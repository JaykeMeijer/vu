\documentclass[10pt,a4paper]{article}
\usepackage[utf8]{inputenc}
\usepackage[english]{babel}
\usepackage{amsmath}
\usepackage{todonotes}
\usepackage{amsfonts}
\usepackage{amssymb}
\usepackage{graphicx}
\usepackage[all]{xy}
\usepackage{listings}
\usepackage[left=3cm,right=3cm,top=3cm,bottom=3cm]{geometry}
\author{Rik van der Kooij 2526314\\Richard Torenvliet 2526864}
\title{Assignment 1 - Design Document}
\begin{document}
\maketitle

%\section{Introduction}

\section{Coarse Grained List}
\subsection{Implementation}
    The coarse grained list is a linked list with one lock to provide a
    concurrent data structure. Before operating on the list threads have to
    acquire this lock and unlock it afterwards. This forces only one thread
    operating on the list at a given time. A simple pseudocode example of the
    add and remove function looks as follows:

\begin{lstlisting}
    add(e) 
        lock()
        insert(e) // insert the element to the list
        unlock()

    remove(e)
        lock()
        rm(e)     // remove the element from the list
        unlock()
\end{lstlisting}

The insert and rm functions are the equivalent of the sequential add and remove
functions.

\subsection{Performance Hypothesis}
One thread will be able to operate on the list at the time. Other threads will
be stuck at acquiring the lock. Increasing the number of elements will increase
the lookup time. If the operating time increases so does the waiting time of
all other threads. More elements should therefore decrease the performance of
the coarse grained list.

Locking once is cheap. If there are no threads that want to operate on the list
simultaneously the performance will be high; one lock operation per add or
remove operation.  With high amount of threads the chance of simultaneous
arrival will be high. Since arrived thread $n$ has to wait for $n-1$ threads
waiting times are long. For high amount of threads the performance will
therefore decrease. 

\section{Coarse Grained Tree} \subsection{Implementation} The coarse grained
tree is a binary tree which uses one lock to provide a concurrent data
structure. Threads operating the tree will have to acquire this lock before
they can do any operations on it. A pseudocode example would look the same as
with the coarse grained list. Although the insert and rm function would be
implemented for a binary tree.

\subsection{Performance Hypothesis} The same performance hypotheses for the
coarse grained list holds for the coarse grained tree. The difference in
performance lies in the data structure. The lookup on the list is of complexity
$O(n)$. A balanced tree has a lookup complexity of $O(log(n))$. The assignment
didn't specify the tree to be balanced. The worst case lookup on the tree is
therefore still $O(n)$, but it should have better average performance.

\section{Fine Grained List} 
\subsection{Implementation}
The fine grained list uses one lock per list element. Before using an element the corresponding lock should be acquired.
Locks of an element cannot be stored in the element itself. This would require us to read the element
before locking. Maurice Herlihy and Nir Shavit solve this problem by adding sentinel head and tail elements.

Inserting and deleting will be done by going through the list locking 2 elements at the time. 
After finding the right element pointers will be set and locks freed. Figure \ref{fig:fine_insert} shows an
example of how an insertion would be executed.

\begin{figure}[h] 
\centerline{ 
    \xymatrix{ 1: &*+++[F=]{head} \ar[r] & *+++[F]{20} \ar[r] & *+++[F]{28} \ar[r] & *+++[F]{tail}\\
               2: &*+++[F=]{head} \ar[r] & *+++[F=]{20} \ar[r] & *+++[F]{28} \ar[r] & *+++[F]{tail}\\
               3: &*+++[F]{head} \ar[r] & *+++[F=]{20} \ar[r] & *+++[F]{28} \ar[r] & *+++[F]{tail}\\
               4: &*+++[F]{head} \ar[r] & *+++[F=]{20} \ar[r] & *+++[F=]{28} \ar[r] & *+++[F]{tail}\\
               5: &*+++[F]{head} \ar[r] & *+++[F=]{20} \ar[r]& *+++[F]{23} \ar[r] & *+++[F=]{28} \ar[r] & *+++[F]{tail}\\
               6: &*+++[F]{head} \ar[r] & *+++[F]{20} \ar[r]& *+++[F]{23} \ar[r] & *+++[F]{28} \ar[r] & *+++[F]{tail}\\
           }}
    \caption{Insert example on fine grained list}
    \label{fig:fine_insert}
\end{figure}

Slow threads won't prevent all other threads from operating. If thread X locks 
item 100 in the list another thread Y can lock all elements 1 to 99. This allows thread Y
to operate on those elements.

\subsection{Performance Hypothesis}
If the amount of elements in the data structure is small, a small part of the list can be accessed by other threads.
As the list grows, more items can be inserted or deleted simultaneously. If the
list contains more items the threads will have a greater working area, since
the locks are per item and not on the data structure itself. So the amount of
elements that can be inserted/deleted per time unit grows. The hypothesis is
that the performance of the list increases when number of items in the list
rises.

The amount of work performed by a thread variates. It depends on the length of the list and how many lookups have to be
performed before deletion. For every lookup, at least two locks and unlocks has
to be performed. This can influence the performance in a negative way. In
worst-case the elements that are added to the list are already sorted, this
means N lookups have to be performed for a list of N items, since they are
added to the end of the list. Locking will become a bottleneck when small
amount of threads want to operate at the same.

\section{Fine Grained Tree}
The fine grained tree also uses an lock for every element. 
The locks are obtained in the same manner as by the fine 
grained list. But now the next Node will be the left or right neighbour, 
so in fact the lookup changes. 

Locks will be acquired in a downward fashion. It is therefore not possible for
deadlock to occur without threads crashes.

\subsection{Performance Hypothesis}
Threads will not block progress for all
other items in the binary tree. This means concurrent progress is possible on
different parts of the tree.  The hypothesis is that when the amount of items
in the binary tree increases, more and more threads can work at the same time
on different parts of the tree and therefore the performance of the tree
increases. Because of the fact that more parts of the data structure can be
accessed by threads the concurrency rises and thereby the amount of items that
can be added per time unit increases.

%\todo[inline]{something about deletion performance, more locks etc.}

\section{Lock Free List}
\subsection{Implementation}
Lock free lists can be implemented with an object that holds a reference to a node in the list and a
Boolean ``mark''. The reference to the item is used to point to the next node,
and the mark is used for marking an object for deletion. These fields can be
atomically updated in the object. Marking objects for deletion fixes the
problem of deleting a reference while another thread uses that node to add a
new node to the list. Now, other threads that perform the add() or remove()
function will remove nodes that are marked. Because the add() and remove()
functions both delete all the marked nodes, an abstraction can be made. A so
called Window() object. This object can carry out this functionality and it
only retrieves the pred and curr when all the objects that are marked for deletion,
prior to the node of interest, are removed. The remove() uses the
pred and curr to mark the curr for deletion(atomically) and the add() function
uses pred and curr to insert a new node between them.

\subsection{Performance Hypothesis}
The lock free list should perform better than the fine grained list. Where slow threads
could hold up other threads at the fine grained list, the lock free list does not support 
this behavior. This is because the add and remove functions are lock-free.

Lookups on the lock free list are very cheap since it is wait-free. The bottleneck of the 
algorithm is when a thread has to retry. This happens when multiple threads want to operating the same element.
Increasing the amount of threads increases the chance to threads want to operate an element at the same time.
The same way, if the amount of elements increases the chance of two threads arriving
at the same element will be lower. Our hypothesis will be that the 
performance will be high when the of elements will be high compared to the
amount of threads.

\section{Lock Free Tree}
\subsection{Implementation}
The lock free tree uses compare and swap operations. Insert and delete
operations on the tree require only a single child pointer change. Using a single CAS operation
would fail with concurrent updates. The implementation uses therefore 3 CAS operations for an insert and
4 for a delete. The idea is that when a CAS operation succeeds it stores enough information
for other threads to complete the other CAS operations. This way a thread stuck at a marked node can help other
threads and itself by performing the other threads operation. This enforces lock-freeness, but since
a process could be doing someone else's work for ever it's not wait-free.


\subsection{Performance Hypothesis}
Since no thread is ever locked the performance should be better than with the fine grained tree. A slow
thread will not slow down the performance since another thread can perform the work instead.

When the amount of threads increases the chance of multiple operations on elements increases as well. Since
threads help each other out a slow thread successfully completing the first CAS operation does not make other
threads wait. This is different from the lock free list, where threads would have to retry. Our hypotheses will be that
with enough threads running this will give very high performances. Enough threads are needed because with no concurrent
threads operating the overhead of the algorithm can still be a bottleneck.

Increasing the amount of elements of the list decreases the change of threads operating on the same element.
While this does not decrease or increase the amount of operations on the tree it will increase the time each
operation will take since less helping will be done.



\end{document}
