\documentclass[10pt,a4paper]{article}
\usepackage[utf8]{inputenc}
\usepackage[english]{babel}
\usepackage{amsmath}
\usepackage{todonotes}
\usepackage{amsfonts}
\usepackage{amssymb}
\usepackage{url}
\usepackage{graphicx}
\usepackage[all]{xy}
\usepackage{listings}
\lstset{ %
  language=Java,                % the language of the code
  basicstyle=\footnotesize,           % the size of the fonts that are used for the code
  numbers=left,                   % where to put the line-numbers
  numberstyle=\tiny\color{gray},  % the style that is used for the line-numbers
  stepnumber=1,                   % the step between two line-numbers. If it's 1, each line 
                                  % will be numbered
  numbersep=5pt,                  % how far the line-numbers are from the code
  backgroundcolor=\color{white},      % choose the background color. You must add \usepackage{color}
  showspaces=false,               % show spaces adding particular underscores
  showstringspaces=false,         % underline spaces within strings
  showtabs=false,                 % show tabs within strings adding particular underscores	
  rulecolor=\color{black},        % if not set, the frame-color may be changed on line-breaks within not-black text (e.g. commens (green here))
  tabsize=2,
  captionpos=b,                   % sets the caption-position to bottom
  breaklines=true,                % sets automatic line breaking
  breakatwhitespace=false,        % sets if automatic breaks should only happen at whitespace
  title=\lstname,                   % show the filename of files included with \lstinputlisting;
                                  % also try caption instead of title
  keywordstyle=\color{blue},          % keyword style
  stringstyle=\color{mauve},         % string literal style
  escapeinside={\%*}{*)},            % if you want to add LaTeX within you code
}
\usepackage[left=3cm,right=3cm,top=3cm,bottom=3cm]{geometry}
\author{Rik van der Kooij: stdnummer, 2526314, VU-net-id: rkj800\\ \&  \\ Richard Torenvliet: stdnummer, 2526864 \ VU-net-id: rtt210}
\title{Concurrency \& Multithreading, concurrent datastructures}
\begin{document}
\maketitle
\tableofcontents

\section{Introduction}
This document is a follow up of the design document made earlier in this coarse. In the design document the idea of different types of synchronizing a concurrent datastructure where discussed and a hypotheses was made about the performance. In the following sections these hypotheses are proofen or disprooven.

\section{Testing}
The datastructures are tested with the shell file: \texttt{bin/test\_data\_structures.sh}. This script takes four parameters where the datastructures can be tested on.

\begin{figure}[h!]
\center
\begin{lstlisting}
test_data_structures <data_structure> <nrThreads> <nrItems> <workTime>
  where:
    <data_structure> in {cgl, cgt, fgl, fgt, lfl, lft}
    <nrThreads> is a number > 0
    <nrItems> is a number > 0
    <workTime> is a number >= 0 (micro seconds)
\end{lstlisting}
\caption{Executing data\_structures.sh}
\end{figure}
It will be interesting to see if the number of threads influences the performance of a datastructure if the number of elements stays equal. Also, the other way around will also be interesting to determine the best concurrent datastructure.

The parameters threads, nritems should be clear. But the workTime is somewhat unclear, but it is a number in microseconds a thread waits before adding the next element. Testing can be done by changing this number. If 0 the threads do not wait before adding the next element what means that the contention is high, so how does the datastructure behave is an interesting question.

With all the datastructures these parameters are changed in order to draw conclusions about the performance. 
\\
\\
The datastructes are tested by doing the following tests \\
\textbf{Test 1:} \\
Changing from 1 thread to 12 threads with 10000 elements, and using 30ms of 
worktime. The datastructures are tested.
\vspace{0.1cm}
\hrule
\vspace{0.1cm}

\begin{lstlisting}
for nr_of_threads in 1 to 12:
    for datastructure in (cgl cgt fgl fgt lfl lft):
        test_data_structures $datastructure $nr_of_threads 10000 30  
\end{lstlisting}


\textbf{Test 2:} \\
Changing from 1 thread to 12 threads with 10000 elements, and using 30ms of 
worktime. The datastructures are tested.
\vspace{0.1cm}
\hrule
\vspace{0.1cm}
\begin{lstlisting}
for num_of_elem in ( 10000 20000 30000 40000 50000 60000 70000 80000 90000 100000):
    for datastructure in (cgl cgt fgl fgt lfl lft):
        bin/test_data_structures $data 4 $num_of_elem 30
\end{lstlisting}

\textbf{Test 3:} \\
Using 4 threads for every datastructure, updating 1000 elements, and altering the workTime from 0 to 80 will give us a view on how the datastructures will behave with different contentions. 
\vspace{0.1cm}
\hrule
\vspace{0.1cm}
\begin{lstlisting}
for worktime in ( 0 10 20 30 40 50 60 70 80 )
    for datastructure in (cgl cgt fgl fgt lfl lft):
        test_data_structures $datastructure 4 10000 $worktime  
\end{lstlisting}

Next sections will discuss the list and tree datastructures. Because of the fact that the performance of a tree farly outstands the performance of the list the lists are compared to lists and trees to trees. The difference in performance of a list and tree are not interesting, because the performance difference of the locking algorithm can not be measured.

Concluding from the previous, a number of research questions can be created and answered in the following sections. These all concern the performance of the datastructure, so how does a certain parameter influence the performance.
\begin{enumerate}
\item The parameter: Number of Threads 
\item The parameter: Number of Items 
\item The parameter: WorkTime 
\end{enumerate}

\section{List performances}
\subsection{Coarse Grained}
The implementations of the concurrent list data structure all follow the pseudocode examples of the book\footnote{The Art of Multicoreprocessor Programming by Maurice Herliy \& Shavit - Chapter 9 Linked Lists: The Role of Locking}. The sorting by hash values of the examples has been changed to the Comparable<T> interface as the assignment requested. The pseudocode code was also changed to allow double elements.

We make use of two sentinel nodes head and tail. These nodes are always the head and tail. The sentinel nodes are useful to create clear code since we do not have to check for empty lists. These are implemented as Node objects which override the compareTo method to always return -1 and 1 respectively. The debug toString method is nothing fancy. It prints the elements in the order they appear in the list with a space in between. The two sentinel nodes are not printed.

Locking is done as we explained in the design document. A single ReentrantLock is used. It is locked at the start of each add and remove method and unlocked at the end. When the lock is required we can simple traverse the linked list and insert/delete an element at the right place. 

% \begin{figure}
% 	\includegraphics[scale=1]{} 
% \end{figure}
\subsection{Fine Grained}
The Fine Grained List uses a lock for each node. We can use a node if we have acquired lock of its predecessor. This means that using the root node requires us to acquire a lock of an non existing predecessor. This is solved by using sentinel nodes we had in the Coarse Grained implementation. Since the sentinel head node never changes we can use it without locking any node.

Traversing the list consists of getting a lock and releasing a previous lock. When we arrive at the place where an element has to be added or removed we hold on to the two locks. Since no other threads can manipulate these nodes we can safely add a new element between the nodes. Removing a node is also safe because the next pointer of the `furthest' cannot be changed concurrently by other threads.




\subsection{Lock Free}
Adding an element can be broken down to one pointer swap. If this is done atomically adding can then be done lock free. Removing is not that simple. No concurrent removals of two subsequent nodes is allowed. This means that atomically setting the next pointer when removing is not enough. We solve this by first marking the node we are removing. If we then make sure that marked nodes cannot change their next pointer, we safely remove the node. Because we have to mark and set a pointer atomically we use the AtomicMarkedReference class of Java.

The toSring method cannot be used for the lock free data structures when other threads are using the list. When we for instance get to a node it might be deleted concurrently. This means we include a removed Node.


\subsection{Benchmarks}

\section{Tree performances}
\subsection{Coarse Grained}
The Coarse Grained Tree is a BST and is similarly locked as the Coarse Grained list version. To be able to make use of the head an tail nodes as a starting point. The binary tree is initialized in a special way. 

\begin{figure}[h!]
\centerline{
	\xymatrix{  
&  & *+++[F]{tail(sentinal)}  \ar[dl] \ar[dr] \\
& *+++[F]{head(sentinal)} \ar[dr] \ar[dl] & & NULL \\
NULL & &*+++[F=]{TREE} \\
	}
}
\end{figure}

The tail node is always the biggest node in the tree, so no node can every be inserted right in the tree. The head node is smaller than tail so it is placed left in the tree, it is also smaller than every other node that will be inserted. So the tree will be placed right of the node. This special way of 
declaring a head and tail node in the tree saves a lot of overhead. So no
To create a node that is always bigger of smaller than other nodes, lies in the compareTo function. Every node has to have a compareTo function, the special head and tail nodes always returns -1 or 1 respectively.

The add function will first take the root node as a starting point. Locks it, and does the update by adding the new Node to the tree and then unlocks the root.

By adding a node, first the proper empty node needs to be found. While no empty node if found, compare the key with the current node and go left or right, accordingly. 

\begin{figure}
\center
\begin{lstlisting}
/* find a parent for the element */
while(curr != null) {
    if(curr.compareTo(t) <= 0) {
        pred = curr;				// go right
        curr = curr.right;
    }
    else if(curr.compareTo(t) > 0) {
        pred = curr;				// go left
        curr = curr.left;
    }
}
\end{lstlisting}
\label{fig:find}
\end{figure}

We end up with the correct empty node and create a new node to add to the 
database. The only thing that needs to checked if the pred, also parent of the new node, needs to placed left or right of the parent.
\\
This is simply done by the following code.
\begin{lstlisting}
/* add the node to the tree */
if(pred.compareTo(t) <= 0) {
    pred.right = node;
} else {
    pred.left = node;
}
\end{lstlisting}
This ensures that the node is placed in the correct place.
\\
The remove function looks similar in the search for the correct node to remove but it breaks out of the while loop, see fig \ref{find}, if it has found the correct node. Now something special happens when removing the node. Three cases can happen in a tree. Either the node to be removed (1) has no children, (2) has one child, (3) has two children. Case (3) is the most difficult one.

In the case the node has no children, the correct node has to be deleted.
\begin{lstlisting}
/* delete left or right child */
if(parent.left == node) 
    parent.left = null;
else
    parent.right = null;
\end{lstlisting}

In case of (2), swap the node with its child. With the left or right node accordingly.
\begin{lstlisting}
if(nodeToDelete.left != null)
    if(parent.left == node) {
        parent.left = node.left;
    } else
        parent.right = node.left;
else 
    if(parent.left == node)
        parent.left = node.right;
    else
        parent.right = node.right;
\end{lstlisting}

In case of (3) the in order-successor needs to found. This is done by traversing one child to the right and traversing al the way to the left to the last node. This node is used to replace the node to be deleted.

\begin{lstlisting}
Node pred = node;
node = node.right; // one step to the right

/* traverse left until node == null */
while(node.left != null) {
    pred = node;
    node = node.left; // go left
}
/* return the parent of that node */
return pred;
\end{lstlisting}

If the inorder successor is the same next right node, the parent is the same as the parent of the in order succesor. So thats why there are two cases. The item of the node is replaced by item of the in-order successor. The in-order-successor's 
\begin{lstlisting}
par_succ = Find_parSucc(node);
if(par_succ == node) {
    /* successor is right child of the node */
    succ = par_succ.right;
    node.item = succ.item;
    node.right = succ.right;
} else {
    /* successor is left somewhere deeper in the tree */
    succ = par_succ.left;
    node.item = succ.item;
    par_succ.left = succ.right;
}
\end{lstlisting}


\subsection{Fine Grained}
The nodes in the fine grained BST the locks are also obtained in a hand-over-hand fashion just like the list. The sentinal nodes are allocated in the same way as the Coarse Grained Tree. With a tailnode at the root, and a head node left, and on the right side of the head will be the tree itself. 

The add function is almost identical as the one in the coarse grained tree accept the locks are differently placed.

Begin by locking the root and the root.left node. Call these pred and curr respectivelly

\begin{lstlisting}
while(curr != null) {
if(curr.compareTo(t) < 0) {
    pred.unlock();
    pred = curr;
    curr = curr.right;
    curr.lock();
} else if(curr.compareTo(t) > 0) {  // maybe just else
    pred.unlock();
    pred = curr;
    curr = curr.left;
    curr.lock();
}
\end{lstlisting}

The other difference is finding the in-order successor, case(3). Again in a hand-over-hand locking the parent of the node of the in-order-successor is found and returned. While holding the lock, it may not be changed in the mean time by another thread.

\begin{lstlisting}
Node pred = node;
node = node.right;
node.lock();
try {
    while(node.left != null) {
        pred = node;
        node = node.left;
        node.lock();
        pred.unlock();
    }
\end{lstlisting}

\subsection{Lock Free}
The Lock Free Tree is implemented as described in the paper Non-blocking Binary Search Tree\footnote{Non-blocking Binary Search Trees by Faith Ellen and Panagiota Fatourou and Eric Ruppert and Franck van Breugel}. 

The lock free tree uses a leaf based tree. Adding a node will always add a leaf or a leaf and internal node. Removing will always remove a leaf and an internal node. This means that we can add a node to the lock free in 3 atomic steps and removing it in 4 atomic steps. The implementation makes uses of flags. Whenever an add or remove encounters a node with a flag it first tries to help it before trying its own operation. This can be done because every step stores enough information to finish the task.

Adding a node to the tree starts with finding which node to add it to, its parent. This node will be marked with an insert flag. We can than safely insert the node to the tree as a leaf and unmark the parent. 

Removing a nodes also starts with finding the node which has to be removed. We then mark the grandparent with a delete flag since one of its child pointers is going to change. The parent is marked with a marked flag which shows that it is going to be removed as well. The removing is then done in the grandparent by setting the appropriate child to the sibling of the removed node. We finish by unmarking the grandparent which allows new operations on the node.

For the toString method it's the same for the lock free tree. Elements we print might already be deleted or we might skip new elements which are added. We are therefore never sure the printed tree has ever really existed.

%TODO de Compare and Swap -> Compare and Set uitleg
The pseudocode description in the paper uses compare and swap methods which return the old value of the CAS object. Java does not support this operation. We can only use compare and set methods. Compare and swap operations where the result is not used have the same functionality as a compare and set. There are however three compare and swap calls in the pseudocode which use the result.

The CAS on line 56 checks whether the compare and swap succeeded by checking if result (old value of the CAS object) had the value it expected. If it did not have the expected value it tries help the result. Our compare and set operation returns a boolean whether or not it succeeded. We can use this instead of comparing the result with our expected value. As far as getting the old value, we just use the CAS object. This can either be the real old value (all is fine) or a new value which a thread has set between our compare and set and our `get'. We try to help the new value, but this not a problem. Helping clean nodes will do nothing and helping a marked (either insert, delete or marked flag) node will only add an additional help call.

The same proof can be given for the compare and swaps on line 81 and 91 which use the result value.





\subsection{Benchmarks}
Due to time issues we were unable to perform sufficient benchmarks and evaluate. We included some graphs of the three benchmarks discussed in section 2. The graphs show however very unreliable values due to the number of benchmarks we performed.


\end{document}
