\documentclass[10pt,a4paper]{article}
\usepackage[utf8]{inputenc}
\usepackage[english]{babel}
\usepackage{amsmath}
\usepackage{todonotes}
\usepackage{amsfonts}
\usepackage{amssymb}
\usepackage{url}
\usepackage{graphicx}
\usepackage[all]{xy}
\usepackage{listings}
\usepackage[left=3cm,right=3cm,top=3cm,bottom=3cm]{geometry}
\author{Rik van der Kooij: stdnummer, 2526314, VU-net-id: rkj800\\ \&  \\ Richard Torenvliet: stdnummer, 2526864 \ VU-net-id: rtt210}
\title{Concurrency \& Multithreading, concurrent datastructures}
\begin{document}
\maketitle
\tableofcontents

\section{Introduction}
This document is a follow up of the design document made earlier in this coarse. In the design document the idea of different types of synchronizing a concurrent datastructure where discussed and a hypotheses was made about the performance. In the following sections these hypotheses are proofen or disprooven.

\section{Testing}
The datastructures are tested with the shell file: \texttt{bin/test\_data\_structures.sh}. This script takes four parameters where the datastructes can be tested on.

\begin{figure}[h!]
\center
\begin{lstlisting}
test_data_structures <data_structure> <nrThreads> <nrItems> <workTime>
  where:
    <data_structure> in {cgl, cgt, fgl, fgt, lfl, lft}
    <nrThreads> is a number > 0
    <nrItems> is a number > 0
    <workTime> is a number >= 0 (micro seconds)
\end{lstlisting}
\caption{Executing data\_structures.sh}
\end{figure}

With all the datastructures these parameters are changed in order to draw conclusions about the performance. 

Every data structure is called in the following ranges.
\begin{lstlisting}
<cgl, cgt, fgl, fgt, lfl, lft
	nrThreads: 1, 2, 4, 8
	nrItems: 100, 1000, 10000, 100000, 1000000
	worktTime: 0
\end{lstlisting}

The number of threads, nritems are clear. The worktime is a number in microseconds a thread waits before adding the next element. If this number is 0 the threads do not wait before adding the next element what means that the contention is high. If the workload increases, the of chance of high contention decreases.

Next sections will discuss the list and tree datastructures. Because of the fact that the performance of a tree farly outstands the performance of the list.

\section{List performances}
\subsection{Coarse Grained}
The implementations of the concurrent list data structure all follow the pseudocode examples of the book\footnote{The Art of Multicoreprocessor Programming by Maurice Herliy \& Shavit - Chapter 9 Linked Lists: The Role of Locking}. The sorting by hash values of the examples has been changed to the Comparable<T> interface as the assignment requested. The pseudocode code was also changed to allow double elements.

We make use of two sentinel nodes head and tail. These are implemented as Node objects which override the compareTo method to always return -1 and 1 respectively. The debug toString method is nothing fancy. It prints the elements in the order they appear in the list with a space in between. The two sentinel nodes are not printed.

Locking is done as we explained in the design document. A single ReentrantLock is used. It is locked at the start of each add and remove method and unlocked at the end.

% \begin{figure}
% 	\includegraphics[scale=1]{} 
% \end{figure}
\subsection{Fine Grained}

\subsection{Lock Free}

\section{Tree performances}
\subsection{Coarse Grained}
The Coarse Grained Tree is a BST and is similarly locked as the Coarse Grained list version. To be able to make use of the head an tail nodes as starting 
\subsection{Fine Grained}

\subsection{Lock Free}

\subsection{conclusion}

\end{document}
